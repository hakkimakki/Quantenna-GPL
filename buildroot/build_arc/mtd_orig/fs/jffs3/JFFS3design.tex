%
% JFFS3 High-level design
%
% Copyright (C) 2005 Artem B. Bityuckiy, dedekind@infradead.org
%
% $Id: JFFS3design.tex,v 1.2 2005/01/22 14:04:43 dedekind Exp $
%

\documentclass[12pt,a4paper,twoside,titlepage]{article}

\begin{document}

%
%  TITLE PAGE
%
\title{JFFS3 design issues}
% Feel free to add your name if you contribute to the document
\author{Artem B. Bityuckiy}
\maketitle

%
% ABSTRACT
%
\thispagestyle{empty}
\begin{abstract}
This document contains different JFFS3 design aspects discussions.
This document tries to define standard JFFS3 dicitionary and terms.
\end{abstract}

%
% COPYRIGHT
%
\thispagestyle{empty}
\section*{License}
TODO: insert GNU's one

%
% ABBREVIATIONS
%
\section*{Abbreviations}
\begin{tabular}{ll}

\large \textbf{Abbrev.}
&
\large \textbf{Definition}
\\[7pt]

\emph{CRC}
&
Cyclic Redurancy Check
\\[4pt]

\emph{JFFS2}
&
Jouranlling Flash File System version 2
\\[4pt]

\emph{JFFS3}
&
Jouranlling Flash File System version 3
\\[4pt]

\end{tabular}

%
% DEFINITIONS
%
\section*{Definitions}
\begin{tabular}{p{3cm}p{11cm}}

\large \textbf{Term}
&
\large \textbf{Definition}
\\[7pt]

\raggedright \emph{Block, Sector} 
&
The minimal flash erasable unit. \\[4pt]

\raggedright \emph{Eblock}
&
JFFS3 may treat several sectors as one eblock. Thus, the minimal
erasable flash unit from the JFFS3's viewpoint is eblock which consists
of one or more flash blocks. \\[4pt]

\raggedright \emph{Node} 
&
Basic JFFS3 data structure - anything which JFFS3 stores on flash is
stored in form of node. There are different types of nodes defined.
Nodes have standard layout - common header, header and
data. Common header is the same for any node type. Header and
data sections are different for different node types.
\\[4pt]

\raggedright \emph{Node common header} 
&
Any node begins with common header which contains node magic
bitmask, node lenght and node type.
\\[4pt]

\raggedright \emph{Node header}
&
The node section next after common header. Unique for each node type.
\\[4pt]
	
\raggedright \emph{Node data}
&
The node section next after node header. Unique for each node type
(e.g., the peace of inode data for the inode nodes, the direntry name in
case of direntry node).
\\[4pt]

\raggedright \emph{Inode node}
&
Any inode in JFFS3 is represented by one (for directory inodes) or more
(for file inodes) inode nodes. Inode nodes are
\texttt{struct jffs3\_raw\_inode} objects.
\\[4pt]

\raggedright \emph{Direntry node}
&
Each JFFS3 directory entry is represented by the direntry node.
Direntry nodes are \texttt{struct jffs3\_raw\_dirent} objects.

\end{tabular}
\newpage

%
% TABLE OF CONTENTS
%
\thispagestyle{empty}
\tableofcontents
\newpage

%
% CHECK SUMMS
%
\pagenumbering{arabic}
\section{Checksums}
JFFS3 protects all nodes by checksums.
In order to be able to work with different node sections (common header,
header, data) independently, each node section has its own checksum.

In JFFS3 checksums are intended to detect node corruptions. Nodes may
become corrupted due to different reasons:
\begin{itemize}
\item flash wearing;
\item external influence (abnormal temperature, radiation, etc);
\item technology specifics, for example:
\begin{itemize}
\item NAND technology implies that block may become bad with very small
possibility.
\item NAND technology implies bits flipping, but this is very rare
phenomena either;
\item AG-AND flash chips impliy rare bit fliping in block if another block
from a specific block group is erased;
\item etc.
\end{itemize}
\end{itemize}

\subsubsection*{JFFS2 correspondence}
JFFS2 uses checksums to detect errors caused by unclean reboots (unclean
reboot may happen any time and partially written nodes may
appear on the flash). JFFS2 treats any checksum error as an error caused by
unclean reboots and have no general ability to distinguish between
checksum errors due to physical flash corruptiona and due to unclean
reboots.

In general, in JFFS3 checksum is not intended to detect errors cased by unclean
reboots. There is another mechanism exists for this purpose. 

\subsection{Checksum modes}
Two JFFS3 working modes are defined:
\begin{enumerate}
\item strict mode;
\item relaxed mode;
\end{enumerate}

\end{document}


